\newpage
\part{Seguridad inform\'{a}tica.}

Es la disciplina encargada de dise\~{n}ar e implementar en lsos sistemas de informaci\'{o}n, las medidas de seguridaden funci\'{o}n del estudio de las amenazas de seguridad y sus costos potenciales con el objetivo de minizar el costo total de \'{e}stos.\\

Medidas de seguridad caracter\'{i}sticas de un sistema orientadas a preventir fallos de seguridad.\\

Fallo de seguridad, divulgaci\'{o}n, creaci\'{o}n, cambio, borrado, denegaci\'{o}n de acceso fuera de los requerimientos del sistema.\\

Las medidas orientadas a resistir y minimizar el impacto o da\~{n}o de los propios fallos ayudando a reestableces el servicio y actualizando las propias medidas de seguridad.

Amenaza de seguridad. Acci\'{o}n potencial de un agente que violente las medidas de seguridad.\\

\'{I}ndice de seguridad. Evento en el que se violan las medidas de seguridad que genera un costo para la instituci\'{o}n que lo sufre o al entorno.\\

Cyber. Se refiere a las medidas reactivas y preventivas contra amenazas de seguridad espec\'{i}ficas.

Objetivos particulares.\\

Confidencialidad - Acceso a la informaci\'{o}n seg\'{u}n pol\'{i}ticas.\\
Integridad - Modificaci\'{o}n y constancia.\\
Disponibilidad - Se puede usar y cu\'{a}ndo se debe usar.\\
Privacidad - Protocolo de los intereses del usuario.\\
Rentabilidad - Retorno de Inversi\'{o}n.\\

\chapter{Ciclo de vida seguro}
O desarrollo seguro de software.
La seguridad debe considerarse y actualizarse durante todo el ciclo de vida.

