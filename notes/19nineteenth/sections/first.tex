\newpage
\chapter{Cliente Abierto.}
\begin{itemize}
	\item Cliente.
	\begin{itemize}
		\item Clientes est\'{a}ndar.
		\item Chrome.
		\item Edge.
		\item Mozilla.
		\item Safari.
		\item Cliente enriquecido (Flash, etc).
	\end{itemize}
	\item Servidor.
	\begin{itemize}
		\item Servicios (Apache, IIS).
		\item Servidor.
		\begin{itemize}
			\item Pol\'{i}ticas.
			\item L\'{o}gica.
			\item Def. Interfaz.
			\item Modelo
			\item Vista
			\item Controlador
		\end{itemize}
	\end{itemize}
\end{itemize}
Pocas implementaciones (Apache, IIS)\\
The unicorn project.
The devops handbook
\chapter{Sistemas federados}
Estos reunen sistemas en distintas arquitecturas y desarrollados en momentos distintos.\\

\chapter{SOA}
Service Oriented Arquitechture.
PAra resolver enlaces punto a punto. Estandariza los protocolos de comunicaci\'{o}n.
Y codificando los datos con XML y la definici\'{o}n de servicio en WSLT.\\
Adem\'{a}s se usan los servicios Bus \'{o} concentradores de servicios que act\'{u}an como intermediarios.\\
Dando independencia f\'{i}sica y servicios de agregaci\'{o}n y transformaci\'{o}n de mensajes.\\

\chapter{BPMS}
Business Process Managment System.\\
Esta se apoya en SOA para hacer una representaci\'{o}n formal de los procesos de negocio.(Idealmente estandarizado)\\

\chapter{Sistemas en la nube}
Permiten contratar capacidad sobre demanda, con provisionamiento automatizado y con cobro sobre uso.\\
Esto se implementa con distintos niveles de automatizaci\'{o}n que se conocen como:\\
\begin{itemize}
	\item Hardware as a service. S.O B\'{a}sico, administraci\'{o}n a cargo del usuario, con tarifa plana y l\'{i}mites de recursos.
	\item Software as a service. Se incluyen componentes de software y conectividad de una aplicaci\'{o}n o plataforma predefinida.
	\item Systemas as a service. Tener la construcci\'{o}n, escalamiento, conexi\'{o}n y administraci\'{o}n de un sistema automatizados en un ambiente que se adapte din\'{a}micamente a la carga.
\end{itemize}

