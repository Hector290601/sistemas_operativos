\chapter{Principios}

\begin{itemize}
	\item M\'{i}nimos privilegios.
	\item Seguridad a profundidad (security depth).
\end{itemize}

\chapter{Actores de las amenazas de la seguridad.}

\begin{itemize}
	\item Aficionado.
	\begin{itemize}
		\item Indiv\'{i}duo, alguna habilidad y recursos t\'{e}cnicos, no es por lucro usualmente por curiosidad, estudio o reputaci\'{o}n.
		\item Es alguien interno a la organizaci\'{o}n.
		\item Es competente.
		\item Suelen ser afables y buenos colaboradores.
		\item Perciben una oportunidad, racionalizaci\'{o}n y forma de eludir represalias.
	\end{itemize}
	\item Criminales inform\'{a}ticos.
	\begin{itemize}
		\item Grupos peque\~{n}os a medianos con un enfoque de lucro.
		\item Grado decente de capacidad t\'{e}cnica.
		\item Suelen ser empresas legalmente establecidas.
	\end{itemize}
	\item Hacktivistas.
	\begin{itemize}
		\item Colectivos de muy diverso tama\~{n}o.
		\item Organizaci\'{o}n vol\'{a}til.
		\item Informal.
		\item Espont\'{a}nea.
		\item Aglomerados por una causa justa y atractiva para la comunidad.
	\end{itemize}
	\item Cyberterroristas.
	\begin{itemize}
		\item Movimientos pol\'{i}ticos o religiosos.
		\item Dedican recursos para financiar ataques de seguridad en favor de su agenda pol\'{i}tica.
	\end{itemize}
	\item Organizaciones financiadas por el estado.
	\begin{itemize}
		\item Grupos gubernamentales dedicados a tareas de "seguridad inform\'{a}tica" con operaciones agresivas ante sistemas y organizaciones detectados como objetivos.
	\end{itemize}
\end{itemize}

\chapter{Criptograf\'{i}a.}

Inicia desdre que se tienen mecanismos f\'{i}siscos 

