\newpage
\chapter{Death locks.}

\textit{Bloqueo mutuo}\\
Es un evento en el que dos o mas procesos se bloquean en espera de dos o mas recursos de modo que no pueden prograsar independientemente del tiempo que pase..\\
Es como cuando dos camiones quieren dar vuelta y se bloquean mutuamente.\\

Tenemos un deadlock solo cuando se cumplen las siguientes 4 condiciones:
\begin{enumerate}
	\item Exclusi\'{o}n mutua. Los procesos reclaman control exclusivo de los procesos.
	\item Hold \& Wait. Los procesos mantienen las asignaciones a los recursos en lo que esperan por los demas recursos que necesitan.
	\item Non-preemption. La asignacion de los recursos no puede ser revocada.
	\item Espera circular. Se genera una secuenia de procesos en espera de recursos asignados a otros procesos.
\end{enumerate}
Estas condiciones sirven para saber cuando es un bloqueo 'normal' o cuando es un deathlock

\section{Mecanismos de detecci\'{o}n}

\subsection{Algoritmo del avestruz}
En sistemas operativos y no implementa ning\'{u}n mecanismo dejando a cargo del usuario el identificar bloqueos y tomar las medidas correctivas.\\
\subsection{Modelo gr\'{a}fico.}
En este se dibuja un grafo en el que se pueden buscar los ciclos.\\
\subsection{An\'{a}lisis matricial.}
Se pueden construir matrices que representen los grafos de asignacion de recuros que se operan para detectar ciclos.\\
Es costos en CPU y solamente permite detectar.
\subsection{An\'{a}lisis de regiones de riesgo.}
Se define como regi\'{o}n de riesgo el estado de una matriz de asignacion de recursos inmediata anterior al deathlock y se niegan las agignaciones de recursos que llevan a regiones de riesgo.\\

\section{Prevencion de los deathlocks.}
Para estos, basta negar alguna condicion del DL.

\subsection{Exclusion mutua.} Reemplazar recursos preemtive por non-preemtive.\\
\subsection{Hold \& Wait} Podemos agrupoar todos los recursos y asignarlos en una sola operacio\'{o}n\\


