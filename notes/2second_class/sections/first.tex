\newpage

\part{\acrlong{firstSection}}
\chapter{Etapa 0}
\section{Computadoras de prop\'{o}sito particular}
Computadoras de prop\'{o}sito particular, ejemplos como la pascalina, que era capaz de calcular integrales a partir de m\'{e}todos num\'{e}ricos\\
Babage nunca construye sus m\'{a}quinas anal\'{i}ticas, pero dej\'{o} muy detallados sus dise\~{n}os e ideas.\\
En la \'{e}poca e Babage no exist\'{i}a el clutch, eran spruckets.\\
En la era moderna, se han constru\'{i}do dos m\'{a}quinas anal\'{i}ticas de Babage, por encargo de Microsoft, una est\'{a} en la colecci\'{o}n privada de Bill Gates, la otra est\'{a} en el museo de la computaci'{o}n, en Sillicon Valley.\\ %% Ver la obra de teatro del censo.
Las tarjetas perforadas modernas tienen 80 columnas, de ah\'{i} sali'{o} la costumbre de dejar el c\'{o}digo en 80 columnas.
Aproximadamente en 1880, se crea un dispositivo mec\'{a}nico para calcular rotores, creando cifrados como el cifrado de C\'{e}sar, \'{e}sta tecnolog\'{i}a se desarrolla en Alemania, en la primera guerra mundial, para la segunda guerra mundial, se comienza a usar la comunicaci\'{o}n radial, usando radio de banda corta.\\
La alemania Nazi comienza a usar las m\'{a}quinas de rotores, con libros de claves diarias, con duraci\'{o}n de entre una semana y un a\~{n}o.\\
A principios de la guerra, la inteliencia Belga captur\'{o} una m\'{a}quina enigma, posteriormente llega Alan Turing a tratar de inverir el proceso de los rotores.\\
En la guerra del pac\'{i}fico se usan las univacs.\\
Desde la \'{e}poca de Napole\'{o}n, acostumbraba apuntar todos los ca\~{n}ones al mismo lugar, para aumentar sus probabilidades de acertar los tiros, cuando la costumbre en la \'{e}poca era dejar que el artillero decidiera el objetivo y disparara a voluntad.\\
En la guerra del pac\'{i}fico se caracterizaba el arma, para poder modelar matem\'{a}ticamente el tiro para la trayectoria deseada, dando manuales para los artilleros donde indicaban c\'{o}mo obtener de manera sencilla las trayectorias.\\
Deciden construir la \textit{Mark II}, la univac usaba bulbos, pero en la guerra del pac\'{i}fico se quedaron sin bulbos, se decide hacer la computadora con relevadores y mandar a gente de la marina a la universidad para poder acelerar los desarrollos, al a\~{n}o la \textit{Mark II}, gener\'{o} las tablas de c\'{a}lculo casi al final de la guerra del pac\'{i}fico.
Luego se decide conectar un equipo telegr\'{a}fico, donado por el tel\'{e}grafo de Hawaii, de tipo tel\'{e}grafo perforado.\\
Con la \textit{Mark II} se inventan las subrutinas y el lenguaje ensamblador.
\paragraph{Grace Cooper} Es quien da la idea de crear lenguajes de programaci\'{o}n modernos, como \textit{COBOL}.\\

\paragraph{Resumen}
\begin{itemize}
	\item Recursos de Hardware justos para cumplir su aplicaci\'{o}n.
	\item Surgen los compliadores (ensambladores) y subrutinas.
	\item Se termina de integrar la memoria funcional y la de datos.
	\item La arquitectura de Vhon Neumann contempla que la memoria de datos y la de almacenamiento sea del mismo tipo.
	\item No hay recursos ni necesidad de sistemas operativos por que se tiene un conocimiento detallado de la arquitectura hecha a la medida.
	\item La computadora de Vhon Neumann genera posibilidades de buffer overflow.
\end{itemize}

\chapter{Etapa 1}
\section{Computadoras de prop\'{o}sito general}
Se busca que las computadoras sean m\'{a}s gen\'{e}ricas para todos los clientes, por que era muy caro dise\~{n}ar m\'{a}quinas espec\'{i}ficas para cada cliente, era mejor una computadora gen\'{e}rica con mejores retornos de inversi\'{o}n.\\
Se ten\'{i}an cientos a miles de Hz de velocidad de procesamiento.\\
Aproximadamente en el 1954-1955\\
Gran desarrollo tecnol\'{o}gico, grandes magnates de los negocios, se plantean vender los equipos a grandes empresas, con muchos trabajadores para solucionar el tema de los c\'{a}lculos de n\'{o}mina.\\
La inversi\'{o}n inicial se recupera en aproximadamente 5 a\~{n}os, el tiempo de vida promedio o esperado de \'{e}stas computadoras era de 10 a\~{n}os, dando aproximandamente 5 a\~{n}os de ganancia pura.\\
Se regalaban el lenguaje, mejoramiento de desarrollos y actualizaciones del equipo.
En el segundo a\~{n}o se buscan nuevas cosas que vender, se comeinzan a vender dispositivos teleequipos m\'{a}s especializados
Ya se tienen los desarrollos realizados, se comienzan a vender equipos actualizados y mejorados, es ah\'{i} cuando surgen las librer\'{i}as de software para acutalizaci\'{o}n y mejora continua de los que se tienen actualmente.\\
Comienzan los archivos de ejecuci\'{o}n por lotes, se regalan los softwares como los Batch, que se encarga de cargar peque\~{n}os programas.
El fabricante de software m\'{a}s importante de la historia es IBM, desde \'{e}stos a\~{n}os (1950s)


\paragraph{Resumen}
\begin{itemize}
	\item Computadoras con recursos  fijos' "sobrados", para una aplicaci\'{o}n.
	\item Se incluyen servicios de desarrollo de programas.
	\item Se generan nuevos lenguajes de progrmaci\'{o}n de prop\'{o}sitp espec\'{i}fico [Cobol, Fortran, etc.]
	\item Se desarrollan dispositivos de I/O progresivamente mejores que requieren librer\'{i}as de software para su uso.
\end{itemize}
\textbf{Aprender PCL}\\
\textbf{Buscar programas de becas de PTC}\\

\chapter{Etapa 2}
Con la adopci\'{o}n de los transistores; con escalas de integraci\'{o}n se tienen capacidades de hardware que crecen exponencialmente (Ley de Moore)
Se comeinza a tener desperdicio de capacidades de computadoras, Paul Art pide dinero a los fabricantes y desarrolladores de software, se realiza la presentaci\'{o}n que se conoce como \textit{La madre de las presentaciones}, propone una interfaz gr\'{a}fica, el mouse, el l\'{a}piz \'{o}ptico, se propone un teclado de combinaciones, co teclados estelogr\'{a}ficos para evitar el movimiento de manos, la presentaci\'{o}n fu\'{e} grabada en cintas de v\'{i}deo de 8mm.
Se proponen novedades, como:
\begin{itemize}
	\item Interfaz gr\'{a}fica.
	\item Programaci\'{o}n orientada a eventos.
	\item Teleconferencias.
	\item Primeras propuestas de internet.
	\item PARC - Estudios de interfaz humana.
\end{itemize}
Se general problemas con el soporte de aplicaciones existentes.\\
La versi\'{o}n est\'{a}ndar de Cobol es el Cobol 86.\\
El mantenimiento de las aplicaciones se complica hasta llegar a la crisiss del sotfware.\\
\begin{center}
	\textit{"Cada bug que corriges genera m\'{a}s bugs que los que resuelve"}
\end{center}


Buscar qu\'{e} son los circuitos supertransistados heterodinos.\\
