\newpage
\part{Cluster.}

Estos son colecciones de computadoras que re\'{u}nen sus recursos para simular un equipo de grandes prestaciones.\\
Para ello, se modifica el S.O para que permita que las tareas se ejecuten en las diversas m\'{a}quinas.

\chapter{Cluster asim\'{e}trico.}

\begin{itemize}
	\item Un nodo administrador o maestro que solo toma roles administrativos y distribuye la carga de trabajo.
	\item M\'{u}ltiples nodos trabajadores que reciben tareas y las ejecutan.
\end{itemize}

\chapter{Cluster asim\'{e}trico.}

\begin{itemize}
	\item Todos los nodos se reparten las tareas de administraci\'{o}n y procesamiento.
	\begin{itemize}
		\item Grid: Varias supercomputadoras comparten todos sus recursos.
	\end{itemize}
\end{itemize}

\newpage

\begin{itemize}
	\item Almacenamiento.
	\item integridad de la informaci\'{o}n.
	\item Pol\'{i}ticas de negocio.
	\item Interfaz de usuario.
	\item Autenticaci\'{o}n.
	\item Protocolo de comunicaci\'{o}n.
\end{itemize}

\part{Migraci\'{o}n de procesos.}

\chapter{RMI}
\textbf{R}emote \textbf{M}ethod \textbf{I}nvocation.\\
Implementa la ejecuci\'{o}n de funciones (o m\'{e}todos) en etuipos remotos (servidores) sin requerir de un API especial o implementaciones complejas.\\
Requieren que se inicie una sesi\'{o}n primero.

\chapter{CORBA}
Common Access Request Broquer.\\
ORB: Procesos encargados del protocolo de comunicaci\'{o}n.

\chapter{RMI Java}
Las clases pueden marcar m\'{e}todos como invocables desde el exterior y se registran en un \'{u}nico directorio por JVM donde cada JVM toma un puerto.

\chapter{Paso de mensajes.}
En estas, las operaciones se entienden como conjuntos de informaci\'{o}n que se generan en un cliente y son enviados (o recibidos) desde un servidor, que los procesa.


