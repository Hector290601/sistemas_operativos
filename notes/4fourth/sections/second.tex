\newpage
\part{Administraci\'{o}n de procesos.}

\chapter{Modelo de procesos.}

\section{Definiciones.}
\begin{itemize}
	\item Programa: Serie de instrucciones ordenadas en un lenguaje regular
	\item Proceso: Instancia de ejecuci\'{o}n de un programa con los siguientes elementos:
		\begin{itemize}
			\item Identificador.
			\item Comportamiento \textit{programa}.
			\item Estado \textit{Almacenado en memoria}.
			\item Hilo de ejecuci\'{o}n \textit{thread}. Es el estado de una secuencia de ejecuci\'{o}n de instrucciones en un proceso. Todos los procesos tienen al menos un hilo.
		\end{itemize}
\end{itemize}
El modelo de computabilidad de \textit{Turing} explica qu\'{e} hacen las computadoras, no lo que se hace con ellas.\\
Los programas que tratan de cubrir todo no deben hacerlo, se deben dividir en diferentes programas que hagan cosas bastante espec\'{i}ficas.\\
No pueden existir procesos sin hilos, los hilos solamente le pertenecen a un proceso.\\

\chapter{Ciclo de vida de los procesos.}
\begin{itemize}
	\item Creaci\'{o}n de procesos.
	\item Ejecuci\'{o}n.
	\item Terminaci\'{o}n.
\end{itemize}
Comienzan, hacen lo que se requiere que hagan y terminan en alg\'{u}n momento.

\section{Creaci\'{o}n.}

\subsection{Formas de creaci\'{o}n de procesos.}
\begin{itemize}
	\item Ceeados desde otro proceso ya existente.
		\begin{itemize}
			\item Hereda variables de ambiente.
			\item Privilegios.
			\item Registro proceso padre.
			\item Grupos de procesos.
		\end{itemize}
	\item Cuando arranca el sistema operativo, el proceso de \textit{Boot} genera una serie de procesos. (Padre = 0), con privilegios de administrador.
		\begin{itemize}
			\item Login.
			\item \textit{Init.d} (ahora \textit{system.d} en system-basedi OS.)
			\item Gestores de dispositivos.
		\end{itemize}
	\item Por petici\'{o}n de usuario en una interfaz gr\'{a}fica. (En sistemas operativos como Android)
	\item Por generar ambientes de ejecuci\'{o}n distintios para las aplicaciones.
	\item Como parte del procesamiento por lotes (Batch, en el contexto de los Mainframes y sistemas que no son de tiempo compartido).
\end{itemize}
Los procesos pueden cambiar sus variables de ambiente.\\
Los procesos registran cu\'{a}l es su proceso padre.\\
Si un proceso modifica sus variables de ambiente y crea un nuevo proceso, pasas las variables de entorno modificadas.\\
Estos procesos a menudo cambiab su usuario y grupo efectivos para operar con permisos m\'{a}s adecuados.\\

\section{Ejecuci\'{o}n.}

Tres posibles estados:
\begin{itemize}
	\item En verdadera ejecuci\'{o}n.
	\item Bloqueado, el CPU lo ignora completamente.
	\item Listo, el proceso est\'{a} esperando a que el CPU lo atienda.
\end{itemize}
\begin{enumerate}
	\item El proceso inicia su turno de ejecuci\'{o}n.
	\item Se termina el \textit{quantum} de ejecuci\'{o}n y se regresa a la fila de listos.
	\item El proceso en ejecuci\'{o}n se bloquea, por ejemplo, por una operaci\'{o}n de \textit{I/O} o una llamada a sistema.
	\item Se desbloquea el proceso.
	\item Bloquear un proceso que no est\'{a} en ejecuci\'{o}n por el administrados o acci\'{o}n de otro proceso.
\end{enumerate}
