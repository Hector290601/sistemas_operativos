\newpage
\part{\acrlong{firstSection}}

\chapter{C\'{o}mputo ubicuo}

\begin{itemize}
	\item Modularidad del S.O.
	\item Mejoras en administraci\'{o}n de recursos nuevos, como la potencia, procesos, memoria no volatil.
	\item Distribu\'{i}das geogr\'{a}ficamente.
\end{itemize}
M\'{a}s o menos en el 2005 se super\'{o} el n\'{u}mero de computadoras conectadas a internet por celulares.

\chapter{Internet de las cosas \textit{IoT}}

\begin{itemize}
	\item Dispositivos de prop\'{o}sito particular con valores agregados al incluir capacidades de procesamiento y telecomunicaciones en conjunto con aplicaciones distribuidas.
	\item Esto genera requerimientos de actualizaci\'{o}n de escaces de recursos y de aplicaciones ad-hoc.
	\item Se generan nuevos problemas de seguridad sin recursos para atenderlos.
\end{itemize}
Poner todas las opciones en costos, para que les den mayor peso.\\
Los fabricantes no ganan mucho por lote, por lo que no se plantea el soporte a largo plazo.\\
Es un mercado \textit{as is}, pero no se pueden vender artefactos siguiendo esta metodolog\'{i}a, debido a que se tienen que arreglar los problemas de compatibilidad, seguridad y mantener precios competitivos.\\
No se tiene claro ell futuro del \textit{IoT}, debido a que falta solucionar muchos problemas.
