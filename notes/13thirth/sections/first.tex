\chapter{Administraci\'{o}n de sistemas de archivos.}
\part{Definiciones}
\section{Archivo}
Es una abstracci\'{o}n, colecci\'{o}n de datos con contexto particular, identificaci\'{o}n, persistente, prop\'{o}sito particular, con una estructura regular y sigue un c\'{o}digo con reglas regulares de representaci\'{o}n.\\

\section{Directorio.}
Es un listado de archivos que puede construir estructuras jer\'{a}rquicas.\\

\section{Metadatos.}
Son datos que almacenan informaci\'{o}n sobre archivos (atributos).\\

\section{Tipo de archivo.}
Son los archivos que se clasifican seg\'{u}n los mecanismos de implementaci\'{o}n en el sistema de archivos.\\
\begin{itemize}
	\item Archivo Plano - Info de App.
	\item Directorio
	\item Asociado a dispositivo de I/O Bloque (As\'{i}ncronos).
	\item I/O Caracter (S\'{i}ncronos)
\end{itemize}

\part{Objetivos.}
\begin{itemize}
	\item Almacenar grandes vol\'{u}menes de informaci\'{o}n.
	\item Reducir overhead.
	\item Informaci\'{o}n sobrevive a los procesos (persistente).
	\item M\'{u}ltiples procesos pueden acceder a la informaci\'{o}n de forma cconcurrente (Protecci\'{o}n de eventos de carrera).
	\item Proporcionar acceso a la informaci\'{o}n.
		\begin{itemize}
			\item APIs estandarizadas.
			\item Incluir utilidades de mantenimiento.
		\end{itemize}
	\item Control de acceso.
	\item Proteger la integridad de la informaci\'{o}n y se use de forma correcta, Depende de la autenticaci\'{o}n y manejo de privilegios.
\end{itemize}

\part{Organizaci\'{o}n del sistema de archivos.}
Para manejo de los metadatos, tenemos los atributos de archivos.
\begin{itemize}
	\item Nombre: Es una caracter\'{i}stica del directorio asociada a la referencia a este.
	\item Identificador: Es un dato que se asocia de forma biun\'{i}voca al archivo.
	\item Ubicaci\'{o}n del archivo: Referencias a los medios de almacenamiento que permitan recuperar la informaci\'{o}n del archivo.
\end{itemize}
