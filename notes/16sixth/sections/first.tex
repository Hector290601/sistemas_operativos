\newpage
\chapter{FAT}
\begin{itemize}
	\item Solo almacena el ID del bloque.
	\item Dos copias de la FAT.
	\item Si se da\~{n}an ambas, podemos explrorar todos los bloques para reconstruir las secuencias de bloques.
	\item Si no recuperamos directorios, solo tenemos datos sin atributos.
	\item Cadenas perdidas (usualmente inutilizables)
\end{itemize}

\chapter{FAT32}
Incrementa el tama\~{n}o de las direcciones y con ello el tama\~{n}o de la partic\'{o}n y archivo m\'{a}ximo respecto a la FAT16.
\textbf{Ambos carecen de control de acceso.}

\chapter{NTFS}
Agrega privilegios para control de acceso y mueve las copias de la FAT a la parte media de la partici\'{o}n.

\chapter{I-nodos}
Estos son nodos de informaci\'{o}n sobre los archivos (metadatos).

Estos se generan por cada archivo que contienen los atributos y las primeras direcciones.

\section{I-nodos multinivel}
\begin{itemize}
	\item I-nodo directo
	\begin{itemize}
		\item I-nodo indirecto
		\begin{itemize}
			\item I-nodo doble indirecto.
			\begin{itemize}
				\item I-nodo triple indirecto.
				\item Etc.
			\end{itemize}
		\end{itemize}
	\end{itemize}
\end{itemize}
Se usa \'{a}rboles binarios balanceados (B+) (por que son la estructura de datos m\'{a}s eficiente conocida), en sistemas de archivos como XFS (Reisner FS).\\
Optimizados a velocidad y tolerancia a fallos.\\

\section{EXT-4}
Esta usa solo I-nodos directos, limitando el tama\~{n}o m\'{a}ximo de archivo.
(2K: M\'{a}ximo 2GB por archivo) y usa Log file system para tolerancia a fallos.
\begin{itemize}
	\item Seguro: Usa log file system en todos los bloques, lento, pero con una mayor tolerancia a fallos.
	\item Medio (default): Solo usa LFS para metadatos.\\
		Incluso si tratamos de terminar las operaciones de metadatos antes de marcar las versiones como inactivas.
	\item R\'{a}pida: Solo almacena metadatos en \textit{LFS} y manda las operaciones a discreci\'{o}n de dispositivo.
\end{itemize}

\chapter{Log file system}
Log file system o sistemas de bit\'{a}cora.\\
En estos, cuando se alteran los archivos, no se editan los bloques que se tienen, si no que se genera una nueva "versi\'{o}n" y solo se marca como inactiva la versi\'{o}n anterior hasta que se termina de genrar la nueva versi\'{o}n.\\
Esto genera un uso progresivo del espacio de la partici\'{o}n.\\
Para recuperar los bloques de versiones anteriores, se tiene un proceso de baja prioridad (nice) que toma los bloques fragmentados y genera una nueva versi\'{o}n de sus archivos compactando el espacio disponible al inicio de la partici\'{o}n.

\chapter{Protecci\'{o}n de archivos..}
\section{Protecci\'{o}n de la informaci\'{o}n.}

\subsection{Mecanismos.}
\begin{itemize}
	\item Dominio de informaci\'{o}n.\\
		Se centraliza el registro y control de la informaci\'{o}n y solamente se da acceso a un grupo particular.
	\item Niveles de Acceso.\\
	\begin{itemize}
		\item P\'{u}blica.
		\item Privada.
		\item Confidencial.
		\item Secreta.
		\item TOP SECRET.
	\end{itemize}
\end{itemize}

\subsection{Principios.}
\begin{itemize}
	\item M\'{i}nimos privilegios.\\
		Los usuarios deben tener los m\'{i}nimos privilegios necesarios para su labor.
	\item Valor de la informaci\'{o}n.
	\begin{itemize}
		\item Oportuna.
		\item Veraz.
		\item Concreta.
	\end{itemize}
\end{itemize}

\subsection{Sistemas de privilegios}
Se busca definir una serie de privilegios para cada uno de los materiales que se puedan asignar de manera individual por grupo y por individuo.

\newpage
Revolution OS
