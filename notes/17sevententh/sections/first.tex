\newpage
\part{Sistema de archivos}
\chapter{RAID}
ARreglo de Discos Redundante Bajo Costo.\\
Es un Estandard en cuanto a las prestaciones que define diversos niveles (pero no para la implementaci\'{o}n).
\begin{itemize}
	\item 0 - Just a bunch of disks.
	\begin{itemize}
		\item Reune toda la capacidad de almacenamiento de los dispositivos.
		\item No a\~{n}ade desempe\~{n}o.
		\item No a\~{n}ade redundancia.
	\end{itemize}
	\item 1 - Mirror.\\
		Los dispositivos se ordenan por pares y deben tener las mismas caracter\'{i}sticas (No se recomienda que sean del mismo lote), se toma uno como maestro y el otro como esclavo, todas las operaciones se realizan en el maestro y la informaci\'{o}n se replica al esclavo. Si un dispositivo falla, el que est\'{a} \'{i}ntegro pasa a ser el maestro.
	\begin{itemize}
		\item No mejora desempe\~{n}o.
		\item Tiene redundancia completa.
	\end{itemize}
	\item 5 y 6.\\
		Con redundancia por bloque distribuido por dispositivo.
	\item Niveles combinados "+"
\end{itemize}

\section{Hot Swap.}
Capacidad de reemplazar dispositivos sin detener la operaci\'{o}n.
\section{Respaldo.}
Planes de respaldo.
\begin{itemize}
	\item ROI.
	\begin{itemize}
		\item Costo de recuperaci\'{o}n con uso del respaldo.
		\item Costo sin backup.
	\end{itemize}
	\item Se requiere un plan que se debe aplicar de forma consistente.
	\item Debe considerar cosas como el costo de los medios.
	\item Esfuerzo.
	\item Protecci\'{o}n a amenazas del ambiente.
	\item Procedimiento de eliminaci\'{o}n.
\end{itemize}
\part{Temas examen}
\textbf{10 de mayo, 19 a 21}\\
\textit{Todo hasta archivos.}\\

\part{Administraci\'{o}n de dispositivos de entrada y salida.}
\begin{itemize}
	\item Variedad de dispositivos de I/O.
	\item Familias de dispositivos por funci\'{o}n.
	\item Mercado.
	\item Funcionalidad.
	\item Facilitar el desarrollo de aplicaciones con independencia f\'{i}sica.
\end{itemize}

\begin{enumerate}
	\item Apps.
	\item Sw del \'{a}rea del usuario API (XFamilia) Rutinas de manejo de errores.
	\item Administraci\'{o}n de dispositivos de I/O.
	\item Controladores espec\'{i}ficos. (Fabricante)
	\item Abstracci\'{o}n de Hardware.
		\begin{itemize}
			\item Comunicaci\'{o}n.
			\item Sincronizaci\'{o}n.
		\end{itemize}
\end{enumerate}

\chapter{Abstracci\'{o}n de Hardware}
\begin{itemize}
	\item Dispositivos s\'{i}ncronos mapeados a memoria.
	\item Programmed interface. (s\'{i}ncrono).
	\item As\'{i}ncrono.
	\begin{itemize}
		\item L\'{i}neas de interrupci\'{o}n.\\
			IRQ - Interrupt Request Line.
		\item As\'{i}ncronas.
	\end{itemize}
	\item DMA - Direct Memory Access.\\
	Se agrega otro bus para ver qui\'{e}n toma control de la memoria.
\end{itemize}
