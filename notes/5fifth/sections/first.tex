\newpage
\chapter{Condiciones de carrera.}

Modelo de regi\'{o}n cr\'{i}tica.\\
Condiciones de soluci\'{o}n de regi\'{o}n cr\'{i}tica.\\

\begin{itemize}
	\item Dos procesos no pueden ingresar al mismo tiemop a la RC.
	\item No hacer suposiciones sobre el n\'{u}mero o velocidad de procesadores.
	\item Un proceso que no est\'{e} en su RC no puede impedir a otro entrar a la RC.
	\item Ning\'{u}n proceso debe esperar indefinidamente.
\end{itemize}
\begin{lstlisting}[language=C]
while(lock); // bussy waiting
lock = false;
region_critica();
lock = true;
\end{lstlisting}
Soluci\'{o}n de Peterson.\\
SW y algoritmo\\
Basada en una variable de turno y un arreglo de inter\'{e}s.\\
\textbf{Usa busy waitings}\\
\large{TSL (Test Set Lock)}.\\
Es una instrucci\'{o}n de ensamblador que en un solo ciclo de reloj.\\
\begin{itemize}
	\item Establece \textit{Z} seg\'{u}n el n\'{u}mero de variables.
	\item Pone un valor $\neq$ 0 en el n\'{u}mero de variables.
\end{itemize}

\begin{lstlisting}[language=bash ]
ENTER_REGION
	TSL # LOCK
	BNE ENTER_REGION
	RET

LEAVE_REGION
	STORE 0 #LOCK
\end{lstlisting}

\section{Sleep/wakeup}
\begin{lstlisting}[language=C]
Enter_region{
	while(lock){
		sleep();
	}
	return();
}
Leave_region{
	wakeup(x);
}
\end{lstlisting}

\section{Sem\'{a}foros.}\\
Un sem\'{a}foro es un conjunto que contiene.
\begin{itemize}
	\item Una variable entera que llamamos em\'{a}foro.
	\item Dos funciones at\'{o}micas que son:
	\begin{itemize}
		\item UP: Incrementa el valor del sem\'{a}foro y desbloquea todos los procesos que est\'{a}n esperando.
		\item DOWN: Trata de decrementar el sem\'{a}foro y se bloquea si vale 0. Cuando se desbloquea reintenta desde el principio.
	\end{itemize}
\end{itemize}
Cuando usamo sem\'{a}foros para resolver RC el sem\'{a}foro solo teoma valores de 0 y 1, y por conveniencia se renombran UP-LOCK y DOWN-UNLOCK.\\
Esta interfaz se conoce como MUTEX.\\
El est\'{a}ndar que desarroll\'{o} \textit{UNIX} para los sem\'{a}foros, se llama \textit{POSIX}.\\
La biblioteca est\'{a}ndar de \textit{C} para manejo de \textit{POSIX} es \textit{p_thread.h}.\\
\section{Monitores}
Son conjuntos de una variable de control, un conjunto de expresiones que evaluan y escriben estas variables y las modifican.\\
Los lenguajes deben garantizar la exclusi\'{o}n mutua en el uso de los miembros del monitor.\\
Implementaci]'{o}n en C++\\
Se usan etiquetas de compilador para se\~{n}alar los elementos de monitor, inclutendo un mutex.\\

Java.\\
\textit{synchronized}
Es una primitiva que se le aplica a un m\'{e}todo que debe tener solo  la regi\'{o}n cr\'{i}tica.\\
Java garantiza la exclusi\'{o}n mutua a nivel de JVM para todos los hilos que use nese m\'{e}todo.\\

Programa multihilos.\\
Crear dos hilos secundarios con p_threads_create y par\'{a}metros por defectos.\\
Ir al Universum.
Proyecto: Algoritmo de persecuci\'{o}n de un rey y una reyna.\\
\begin{enumerate}
	\item El primer hilo mover\'{a} al rey tratando de alejarse lo m\'{a}s posible de la reyna.
	\item El segundo hilo mueve a la reyna tratando de capturar al rey.
	\item Ambos hilos son concurrentes y no usan turnos, se mueven todas las veces que pueden hasta lograr su objetivo.
	\item Al terminar cualquiera de los dos hilos, se imprime el n\'{u}mero de movimientos de cada pieza y termina el proceso.
	\item El rey y la reyna empiezan donde inician habitualmente, el rey es blancas y la reyna negras.
	\item Ambos comienzan en la posici\'{o}n del rey de su color.
	\item Los dos pueden moverse en cualquiera de las ocho direcciones, el rey solamente una casilla y la reyna las que sean.
\end{enumerate}

