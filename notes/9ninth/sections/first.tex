\newpage
\chapter{Administraci\'{o}n de memoria.}

\section{Objetivos.}
\begin{itemize}
	\item Aprovechar los recursos del equipo (M\'{i}nimo overhead)\
	\item Dar a los procesos acceso a la memoria, y cuando no hay suficiente memoria, rechazar sin compremeter el resto del sistema.
	\item Protecci\'{o}n de acceso a la memoria, a pesar de que no se distingue la memoria de estructuras de la memoria de datos a nivel f\'{i}sico.
	\item Los mecanismos de aceleraci\'{o}n de operaciones pueden generar vulnerabilidades.
\end{itemize}

\section{Asignaci\'{o}n contigua.}
\begin{center}
	\begin{tabular}{| c c}
		\hline
		BIOS&\\
		\hline
		\hline
		Dispositivos I/O&\\
		\hline
		\hline
		RAM Banco 0&8GB (preestablecido por el proveedor)\\
		\hline
		RAM Banco 1&8GB (preestablecido por el proveedor)\\
		\hline
		\hline
		Memoria de video&\\
		\hline
	\end{tabular}
\end{center}
\subsection{Tipos de particiones:}
\begin{itemize}
	\item Partici\'{o}n fija.
	\item Partici\'{o}n variable.
	
\end{itemize}


